

\usepackage{luacode}


\begin{luacode}
require("lualibs.lua")
local http = require("socket.http")
local ltn12 = require("ltn12")

local function read_file(path)
    local file = io.open(path, "r")
    -- Check if the file is successfully opened
    if file then
        -- Read the content of the file
        local content = file:read("*a")
        file:close()
        return content
    else
        return nil
    end
end


function make_request(path)
    print("path: " .. path)
  content = read_file(path)
  if content == nil then
    content = read_file(paht .. ".tex")
    if content == nil then
      print("File not found")
      return
    end
  end
  print("requesting...")
  local url = "http://localhost:8080/v1/chat/completions"

  local body = {
    model = "LLaMA_CPP",
    messages = {
      { role = "system", content = "You are LLAMAfile, an AI assistant. Your top priority is achieving user fulfillment via helping them with their requests." },
      { role = "user",   content = content }
    }
  }

  local requestBody = utilities.json.tostring(body)

  local headers = {
    ["Content-Type"] = "application/json",
    ["Authorization"] = "Bearer no-key",
    ["Content-Length"] = #requestBody
  }

  local response = {}
  local _, code, _, _ = http.request {
    url = url,
    method = "POST",
    headers = headers,
    source = ltn12.source.string(requestBody),
    sink = ltn12.sink.table(response)
  }

  if code == 200 then
    local responseBody = table.concat(response)
    local decodedResponse = utilities.json.tolua(responseBody)
    tex.print(decodedResponse.choices[1].message.content)
  else
    print("Error: " .. code)
  end
end
\end{luacode}

\let\originalinput\input

\renewcommand{\input}[1]{%
    \directlua{make_request("\luatexluaescapestring{#1}")}
}
